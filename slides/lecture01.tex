\input{preamble}

\title{Session I\\Survey Experiments in Context}

\date[]{}

\begin{document}

\frame{\titlepage}


\frame{\tableofcontents}


\frame{

\frametitle{Activity!}

\only<2-4,6>{
\begin{enumerate}
\item<2-4,6> Ask you to guess a number
\item<3-4,6> Number off 1 and 2 across the room
\item<4,6> Group 2, close your eyes
\item<6> Group 1, close your eyes
\end{enumerate}
}

\Large
\only<5>{\textit{Group 1}\\ Think about whether the population of Chicago is more or less than 500,000 people. What do you think the population of Chicago is?}
\only<7>{\textit{Group 2}\\ Think about whether the population of Chicago is more or less than 10,000,000 people. What do you think the population of Chicago is?}

}
\frame{}

\frame{

\frametitle{Enter your data}

\begin{itemize}\itemsep1em
\item Go here: \url{http://bit.ly/297vEdd}
\item Enter your guess and your group number
\end{itemize}

%\url{http://goo.gl/forms/xDW4FLm9pau0O8zz2}

}


\frame{

\frametitle{Results}

\begin{itemize}\itemsep1em
\item True population: 2.79 million
\item<2-> What did you guess? \href{https://docs.google.com/spreadsheets/d/1SKWljS1EeNkAV5V0NZUwrKOu3LQFILVMB37xfTxyrPM/edit?usp=sharing}{(See Responses)}
\item<3-> What's going on here?
	\begin{itemize}
	\item An experiment!
	\item Demonstrates ``anchoring'' heuristic
	\end{itemize}
\item<4-> Experiments are easy to analyze, but only if designed and implemented well
\end{itemize}

}



\section{Introductions}
\frame{\tableofcontents[currentsection]}

\frame{
\frametitle{Who am I?}

\small

\begin{itemize}\itemsep0.25em

\item Thomas Leeper

\item Associate Professor in Political Behaviour at London School of Economics

\begin{itemize}
\item 2013--15: Aarhus University (Denmark)
\item 2008--12: PhD from Northwestern University (Chicago, USA)
\item Birth--2008: Minnesota, USA
\end{itemize}

\item Interested in public opinion and political psychology

\item Email: \href{mailto:t.leeper@lse.ac.uk}{t.leeper@lse.ac.uk}

\end{itemize}

}


\frame{

\frametitle{Who are you?}

\begin{itemize}\itemsep1em

\item Introduce yourself to a neighbour

\item Where are you from?

\item What do you hope to learn from the course?

\end{itemize}

}



\frame{

\frametitle{Quick Survey}

\begin{enumerate}\itemsep0.5em
\item<2-> How many of you have worked with survey data before?
\item<3-> Of those, how many of you have \textit{performed} a survey before?
\item<4-> How many of you have worked with experimental data before?
\item<5-> Of those, how many of you have \textit{performed} an experiment before?
\end{enumerate}

}



\section{Course Outline}
\frame{\tableofcontents[currentsection]}


\frame{

\frametitle{Course Materials}

\begin{center}
All material for the course is available at:\\

\vspace{1em}

\large\url{http://www.thomasleeper.com/surveyexpcourse/}
\end{center}

}

\frame{

\frametitle{Learning Outcomes}

\small

By the end of the week, you should be able to\dots

\begin{enumerate}
\item<2-> Explain how to analyze experiments quantitatively.
\item<3-> Explain how to design experiments that speak to relevant research questions and theories.
\item<4-> Evaluate the uses and limitations of several common survey experimental paradigms.
\item<5-> Identify practical issues that arise in the implementation of experiments and evaluate how to anticipate and respond to them.
\end{enumerate}

}


\frame{

\frametitle{Schedule of Four Sessions}

\begin{enumerate}\itemsep0.5em
\item Survey Experiments in Context
\item Examples and Paradigms
\item Hands-on Session
\item Practical Issues
\end{enumerate}

}


\questions



\section[History/Logic]{History and Logic}
\frame{\tableofcontents[currentsection]}

\frame{

\frametitle{Experiments: History I}

Oxford English Dictionary defines ``experiment'' as:

\begin{enumerate}
\item A scientific procedure undertaken to make a discovery, test a hypothesis, or demonstrate a known fact
\item A course of action tentatively adopted without being sure of the outcome
\end{enumerate}
}

\frame{

\frametitle{Experiments: History II}

\begin{itemize}\itemsep0.75em
\item ``Experiments'' have a very long history

\item Major advances in design and analysis of experiments based on agricultural and later biostatistical research in the 19th century (Fisher, Neyman, Pearson, etc.)

\item<2-> Multiple origins in the social sciences

	\begin{itemize}
	\item<3-> First randomized experiment by Peirce and Jastrow (1884)
	\item<3-> Gosnell (1924)
	\item<3-> LaLonde (1986)
	\item<3-> Gerber and Green (2000)
	\end{itemize}

\end{itemize}

}


\frame{

\frametitle{Experiments: History III}

\small

\begin{itemize}
\item Rise of surveys in the behavioral revolution
	\begin{itemize}
	\item Survey research not heavily experimental because interviewing was mostly paper-based
	\item ``Split ballots'' (e.g., Schuman \& Presser; Bishop)
	\end{itemize}
\item<2-> 1983: Merrill Shanks and the Berkeley Survey Research Center develop CATI
\item<3-> Mid-1980s: Paul Sniderman \& Tom Piazza performed the first \textit{modern} survey experiment\only<3->{\footnote{Sniderman, Paul M., and Thomas Piazza. 1993. \textit{The Scar of Race}. Cambridge, MA: Harvard University Press.}}
	\begin{itemize}\footnotesize
	\item Then: the ``first multi-investigator''
	\item Later: Skip Lupia and Diana Mutz created TESS
	\end{itemize}

\end{itemize}

}

\frame{

\frametitle{TESS}

\small

\begin{itemize}
\item Time-Sharing Experiments for the Social Sciences
\item Multi-disciplinary initiative that provides infrastructure for survey experiments on nationally representative samples of the United States population
\item Great resource for survey experimental materials, designs, and data
\item Funded by the U.S. National Science Foundation
\item Anyone anywhere in the world can apply
\item See also: \href{https://www.lissdata.nl/lissdata/}{LISS}, \href{http://www.uib.no/en/citizen}{Bergen's Citizen Panel}, \href{http://lore.gu.se/surveys/citizen}{Gothenburg's Citizen Panel}
\end{itemize}

}





\frame{

\frametitle{The First Survey Experiment}

Hadley Cantril (1940) asks 3000 Americans either:

\vspace{1em}

\begin{columns}
\begin{column}[t]{.5\textwidth}
\onslide<2->{Do you think the U.S. should do more than it is now doing to help England and France?
\vspace{1em}
\begin{itemize}
\item Yes\onslide<4->{: 13\%}
\item No
\end{itemize}
}
\end{column}
\begin{column}[t]{.5\textwidth}
\onslide<3->{Do you think the U.S. should do more than it is now doing to help England and France \textcolor{red}{in their fight against Hitler}?
\begin{itemize}
\item Yes\onslide<5->{: 22\%}
\item No
\end{itemize}
}
\end{column}
\end{columns}

\vspace{1em}

\onslide<6->{The ``Hitler effect'' was 22\% - 13\% = 9\%}

}



\frame{

\frametitle{Definitions I}

\begin{itemize}\itemsep0.75em
\item<1-> A randomized experiment is:
\end{itemize}
	\begin{quote}\small
		The observation of units after, and possibly before, a randomly assigned intervention in a controlled setting, which tests one or more precise causal expectations
	\end{quote}
\begin{itemize}\itemsep0.75em
\item<2-> If we manipulate the thing we want to know the effect of ($X$), and control (i.e., hold constant) everything we do not want to know the effect of ($Z$), the only thing that can affect the outcome ($Y$) is $X$.
\end{itemize}

}


\frame{
\frametitle{Definitions II}

\small

\begin{itemize}
\item<2-> A survey experiment is just an experiment that occurs in a survey context
	\begin{itemize}
	\item As opposed to in the field or in a laboratory
	\end{itemize}
\item<3-> Can be in any mode (face-to-face, CATI, IVR, CASI, etc.)
\item<4-> May or may not involve a representative population
	\begin{itemize}
	\item Mutz (2011): ``population-based survey experiments''
	\end{itemize}
\end{itemize}
}

\frame{

\frametitle{Definitions II}

\only<2>{\textbf{Unit}: A physical object at a particular point in time}

\only<3>{\textbf{Treatment}: An intervention, whose effect(s) we wish to assess relative to some other (non-)intervention

\vspace{1em}

Synonyms: manipulation, intervention, factor, condition, cell

}

\only<4>{\textbf{Outcome}: The variable we are trying to explain}

\only<5>{
\textbf{Potential outcomes}: The outcome value for each unit that we \textit{would observe} if that unit received each treatment\\

\vspace{1em}

Multiple potential outcomes for each unit, but we only observe one of them

}

\only<6>{\textbf{Causal effect}: The comparisons between the unit-level potential outcomes under each intervention\\

\vspace{1em}

\textit{This is what we want to know!}
}

\only<7>{\textbf{Average causal effect}: Difference in mean outcomes between treatment groups\\

\vspace{1em}

\textit{This is \textit{almost} what we want to know!}
}


}


\frame{

\frametitle{Example}

\onslide<2->{\textbf{Unit}: Americans in 1940}

\onslide<3->{\textbf{Outcome}: Support for military intervention}

\onslide<4->{\textbf{Treatment}: Mentioning Hitler versus not}

\onslide<5->{
\textbf{Potential outcomes}:

\begin{enumerate}
\item Support in ``Hitler'' condition
\item Support in control condition
\end{enumerate}

}

\onslide<6->{\textbf{Causal effect}: Difference in support between the two question wordings for each respondent

\begin{itemize}
\item<7-> Individual treatment effect not observable!
\item<8-> Average effect (ATE) is the mean-difference
\end{itemize}
}

}


\questions


\frame{

\frametitle{Why are experiments useful?}

\begin{center}
\Large
\onslide<2->{Causal inference!}
\end{center}

}


\frame{

\frametitle{Addressing Confounding}

In observational research\dots

\begin{enumerate}\itemsep0.5em
\item<2-> Correlate a ``putative'' cause ($X$) and an outcome ($Y$), where $X$ temporally precedes $Y$
\item<3-> Identify all possible confounds (\textbf{Z})
\item<4-> ``Condition'' on all confounds
	\begin{itemize}
	\item Calculate correlation between $X$ and $Y$ at each combination of levels of \textbf{Z}
	\end{itemize}
\item<5-> Basically: $Y = \beta_0 + \beta_1 X + \beta_{2-k} \mathbf{Z} + \epsilon $
\end{enumerate}

}


\begin{frame}
\begin{center}
\begin{tikzpicture}[>=latex',circ/.style={draw, shape=circle, node distance=5cm, line width=1.5pt}]
    \draw (0,0) node[left, text width=3cm, align=center] (X) {Salience of\\Hitler};
    \draw[->] (X) -- (3,0) node[right, text width=3cm, align=center] (Y) {Support for\\Military\\Intervention};
    \draw (-2,2) node[above, text width=3cm, align=center] (Z) {Media Coverage};
    \draw[->] (Z) -- (Y);
    \draw[->] (3,2) node[above] (A) {Demographics} -- (Y);
    \draw[->] (4,-2) node[below] (B) {Ideology} -- (Y);
    \draw[->] (1,-2) node[below, text width=3cm, align=center] (E) {Political\\Sophistication} -- (Y);
	\draw<2->[->, dashed, very thick] (Z) -- (X);
	\draw<3->[->, dashed, very thick] (E) -- (X);
\end{tikzpicture}
\end{center}
\end{frame}


\frame{

\frametitle{Experiments are different}

\begin{enumerate}\itemsep0.75em
\item<2-> Causal inferences from \textit{design} not \textit{analysis}
\item<3-> Solves both temporal ordering and confounding
	\begin{itemize}
   		\item Treatment ($X$)  applied by researcher before outcome ($Y$)
   		\item Randomization eliminates confounding ($\mathbf{Z}$)
   		\item We don't need to ``control'' for anything
	\end{itemize}
\item<4-> Basically: $Y = \beta_0 + \beta_1 X + \epsilon $
\item<5-> Thus experiments are a ``gold standard''
\end{enumerate}

}


% Mill's method of difference
\frame{
\frametitle{{\normalsize Mill's Method of Difference}}

\small

If an instance in which the phenomenon under investigation occurs, and an instance in which it does not occur, \textbf<2->{have every circumstance save one in common}, that one occurring only in the former; \textbf<2->{the circumstance in which alone the two instances differ, is the} effect, or \textbf<2->{cause}, or an necessary part of the cause, \textbf<2->{of the phenomenon}.
}


\questions


\frame{

\frametitle{Neyman-Rubin Potential Outcomes Framework}

If we are interested in some outcome $Y$, then for every unit $i$, there are numerous ``potential outcomes'' $Y*$ only one of which is visible in a given reality. Comparisons of (partially unobservable) potential outcomes indicate causality.

}

\frame{

\frametitle{Neyman-Rubin Potential Outcomes Framework}

Concisely, we typically discuss two potential outcomes:

\begin{itemize}\small
\item $Y_{0i}$, the \textit{potential} outcome \textit{realized} if $X_i = 0$ (b/c $D_i = 0$, assigned to control)
\item $Y_{1i}$, the \textit{potential} outcome \textit{realized} if $X_i = 1$ (b/c $D_i = 1$, assigned to treatment)
\end{itemize}

}



% design-based experimental inference
\frame{
	\frametitle{Experimental Inference I}
	\small
	\begin{itemize}\itemsep0.5em
    	\item<1-> Each unit has multiple \textit{potential} outcomes, but we only observe one of them, randomly
    	\item<2-> In this sense, we are sampling potential outcomes from each unit's population of potential outcomes
		\only<2->{
			\begin{center}
			\begin{tabular}{ccccc}
			unit & low & high & \onslide<3->{control} & \onslide<4->{etc.} \\ \midrule
			1 & ? & ? & \onslide<3->{?} & \onslide<4->{\dots} \\
			2 & ? & ? & \onslide<3->{?} & \onslide<4->{\dots} \\
			3 & ? & ? & \onslide<3->{?} & \onslide<4->{\dots} \\
			4 & ? & ? & \onslide<3->{?} & \onslide<4->{\dots} \\ \bottomrule
			\end{tabular}
			\end{center}
		}
	\end{itemize}
}

\frame{
	\frametitle{Experimental Inference II}
	\small
	\begin{itemize}\itemsep0.5em
    	\item<1-> We cannot see individual-level causal effects
    	\item<2-> We can see \textit{average causal effects}
    		\begin{itemize}
        		\item<2-> Ex.: Average difference in military support among those thinking of Hitler versus not
    		\end{itemize}
    	\item<3-> We want to know: $TE_i = Y_{1i} - Y_{0i}$
	\end{itemize}
}

\frame{
	\frametitle{Experimental Inference III}
	\small
	\begin{itemize}\itemsep0.5em
		\item<1-> We want to know: $TE_i = Y_{1i} - Y_{0i}$ for every $i$ in the population
		\item<2-> We can average: $E[TE_i] = E[Y_{1i} - Y_{0i}] = E[Y_{1i}] - E[Y_{0i}]$
		\item<3-> But we still only see one potential outcome for each unit:\\ \vspace{1em}
    		$ATE_{naive} = E[Y_{1i} | X = 1] - E[Y_{0i} | X = 0]$
    	\item<4-> Is this what we want to know?
	\end{itemize}
}


\frame{
	\frametitle{Experimental Inference IV}
	\small
	\begin{itemize}\itemsep0.5em
	\item What we want and what we have:
		\begin{align}
		ATE & = E[Y_{1i}] - E[Y_{0i}] \\[1em]
		ATE_{naive} & = E[Y_{1i} | X = 1] - E[Y_{0i} | X = 0]
		\end{align}		
	\item<2-> Are the following statements true?\\
  		\begin{itemize}\itemsep0.5em
      		\item<2-> $E[Y_{1i}] = E[Y_{1i} | X = 1]$
      		\item<2-> $E[Y_{0i}] = E[Y_{0i} | X = 0]$
  		\end{itemize}
  	\item<3-> Not in general!
  	\end{itemize}
}

\frame{
	\frametitle{Experimental Inference V}
	\small
	\begin{itemize}\itemsep0.5em
    	\item Only true when both of the following hold:
    	\begin{align}
    	E[Y_{1i}] = E[Y_{1i} | X = 1] = E[Y_{1i} | X = 0]\\
    	E[Y_{0i}] = E[Y_{0i} | X = 1] = E[Y_{0i} | X = 0]
    	\end{align}
    	\item In that case, potential outcomes are \textit{independent} of treatment assignment
		\item If true (e.g., due to randomization of $X$), then:
    	\begin{align*}
    	ATE_{naive} & = E[Y_{1i} | X = 1] - E[Y_{0i} | X = 0] \tag{5}\\
    	& = E[Y_{1i}] - E[Y_{0i}]\\
    	& = ATE
    	\end{align*}
	\end{itemize}
}

\frame{
	\frametitle{Experimental Inference VI}
	
	\begin{itemize}\itemsep1em
    	\item This holds in experiments because of a \textit{physical process of randomization}\footnote{Random means ``known probability of treatment'' not ``haphazard''.}
   		\item<2-> Units differ only in side of coin that was up
	   		\begin{itemize}\footnotesize
	   		\item $X_i = 1$ only because $D_i = 1$
	   		\end{itemize}
	   	\item<3-> Implications:
		   	\begin{itemize}
		   	\item Covariate balance
		   	\item Potential outcomes balanced and independent of treatment assignment
		   	\item No confounding (selection bias)
		   	\end{itemize}
	\end{itemize}
}




\begin{frame}
\begin{center}
\begin{tikzpicture}[>=latex',circ/.style={draw, shape=circle, node distance=5cm, line width=1.5pt}]
    \draw (0,0) node[left, text width=3cm, align=center] (X) {Salience of\\Hitler};
    \draw[->] (X) -- (3,0) node[right, text width=3cm, align=center] (Y) {Support for\\Military\\Intervention};
    \draw (-2,2) node[above, text width=3cm, align=center] (Z) {Media Coverage};
    \draw[->] (Z) -- (Y);
    \draw[->] (3,2) node[above] (A) {Demographics} -- (Y);
    \draw[->] (4,-2) node[below] (B) {Ideology} -- (Y);
    \draw[->] (1,-2) node[below, text width=3cm, align=center] (E) {Political\\Sophistication} -- (Y);
	\draw[->, dashed] (E) -- (X);
	\draw[->, dashed] (Z) -- (X);
	\draw<2-> (-2,-2) node[left, color=red, text width=2cm, align=center] (T) {Randomly\\Assigned\\Prime};
	\draw<2->[->, very thick, color=red] (T) -- (X);
\end{tikzpicture}
\end{center}
\end{frame}


\questions



\frame{
\frametitle{Experimental Analysis I}
\small
\begin{itemize}
\item The statistic of interest in an experiment is the \textit{sample average treatment effect} (SATE)
\item If our sample is \textit{representative}, then this provides an estimate of the population average treatment (PATE)
	\begin{itemize}
	\item Design-based random sampling
	\item Model-based re-weighting
	\end{itemize}
\item<2-> This boils down to being a mean-difference between two groups:
	\begin{equation}
	SATE = \frac{1}{n_1}\sum Y_{1i} - \frac{1}{n_0}\sum Y_{0i}
	\end{equation}
\end{itemize}
}



\frame[label=tidy]{

\frametitle{Tidy Experimental Data}

An experimental data structure looks like:

\small

\begin{center}
\begin{tabular}{ccc}
\texttt{unit} & \texttt{treatment} & \texttt{outcome} \\ \hline 
1 & 0 & 13 \\
2 & 0 & 6 \\
3 & 0 & 4 \\
4 & 0 & 5 \\
5 & 1 & 3 \\
6 & 1 & 1 \\
7 & 1 & 10 \\
8 & 1 & 9 \\ \hline
\end{tabular}
\end{center}

}

\frame{

\frametitle{Tidy Experimental Data}

Sometimes it looks like this instead, which is bad:

\small

\begin{center}
\begin{tabular}{cccc}
\texttt{unit} & \texttt{treatment} & \texttt{outcome0}  & \texttt{outcome1} \\ \hline 
1 & 0 & 13 & NA \\
2 & 0 & 6 & NA \\
3 & 0 & 4 & NA \\
4 & 0 & 5 & NA \\
5 & 1 & NA & 3 \\
6 & 1 & NA & 1 \\
7 & 1 & NA & 10 \\
8 & 1 & NA & 9 \\ \hline
\end{tabular}
\end{center}

}

\againframe{tidy}


\frame{

\frametitle{Computation of Effects I}

\begin{itemize}\itemsep0.5em
\item In practice we often estimate SATE using t-tests, ANOVA, or OLS regression
\item These are all basically equivalent
\item<2-> Reasons to choose one procedure over another:
	\begin{itemize}
	\item<2-> Disciplinary norms
	\item<3-> Ease of interpretation
	\item<4-> Flexibility for >2 treatment conditions
	\end{itemize}
\end{itemize}

}



\begin{frame}[fragile]

\frametitle{Computation of Effects II}

R:\small
\begin{verbatim}
t.test(outcome ~ treatment, data = data)
lm(outcome ~ factor(treatment), data = data)
\end{verbatim}

\vspace{1em}

Stata:\small
\begin{verbatim}
ttest outcome, by(treatment)
reg outcome i.treatment
\end{verbatim}

\end{frame}


\questions

\frame{
\frametitle{Experimental Analysis II}
	\small
\begin{itemize}\itemsep0.5em
\item We don't just care about the size of the SATE. We also want to know whether it is significantly different from zero (i.e., different from no effect/difference)
\item Thus we need to estimate the \textit{variance} of the SATE
\item The variance is influenced by:
	\begin{itemize}
	\item Total sample size
	\item Element variance of the outcome, $Y$
	\item Relative size of each treatment group
	\item (Some other factors)
	\end{itemize}
\end{itemize}
}


\frame{
\frametitle{Experimental Analysis III}
	\small
\begin{itemize}\itemsep0.5em
\item Formula for the variance of the SATE is:\\

$\widehat{Var}(SATE) = \dfrac{\widehat{Var}(Y_0)}{n_0} + \dfrac{\widehat{Var}(Y_1)}{n_1}$

	\begin{itemize}
	\item $\widehat{Var}(Y_0)$ is control group variance
	\item $\widehat{Var}(Y_1)$ is treatment group variance
	\end{itemize}

\item We often express this as the \textit{standard error} of the estimate:\\
$\widehat{SE}_{SATE} = \sqrt{\frac{\widehat{Var}(Y_0)}{n_0} + \frac{\widehat{Var}(Y_1)}{n_1}}$
\end{itemize}
}


\frame{

\frametitle{Intuition about Variance}

\begin{itemize}\itemsep1em
\item Bigger sample $\rightarrow$ smaller SEs
\item Smaller variance $\rightarrow$ smaller SEs
\item Efficient use of sample size:
	\begin{itemize}
	\item When treatment group variances equal, equal sample sizes are most efficient
	\item When variances differ, sample units are better allocated to the group with higher variance in \emph{Y}
	\end{itemize}
\end{itemize}


}



\frame{

\frametitle{Statistical Power}

\begin{itemize}\itemsep0.5em
\item Power analysis is used to determine sample size before conducting an experiment

\item Type I and Type II Errors

\begin{center}
\begin{tabular}{lcc}
\toprule
& $H_0$ False & $H_0$ True \\ 
& ($|ATE| > 0$) & ($ATE = 0$) \\ \midrule
Reject $H_0$ & \textbf{True positive} & Type I Error \\
Accept $H_0$ & Type II Error & True zero \\ \bottomrule
\end{tabular}
\end{center}

	\begin{itemize}
	\item True positive rate ($1-\kappa$) is power
	\item False positive rate is the significance threshold ($\alpha$)
	\end{itemize}

\end{itemize}
}


\frame{

\frametitle{Doing a Power Analysis}

\begin{itemize}
\item $\mu$, Treatment group mean outcomes
\item $N$, Sample size
\item $\sigma$, Outcome variance
\item $\alpha$ Statistical significance threshold
\item $\phi$, a sampling distribution
\end{itemize}

$Power = \phi\left( \frac{|\mu_1 - \mu_0|\sqrt{N}}{2\sigma} - \phi^{-1}\left( 1 - \frac{\alpha}{2} \right) \right)$

}



\frame{
\frametitle{Intuition about Power}

Minimum detectable effect is the smallest effect we could detect given sample size, ``true'' ATE, variance of outcome measure, power ($1-\kappa$), and $\alpha$.\\

\vspace{1em}

\only<2->{In essence: some non-zero effect sizes are not detectable by a study of a given sample size.}

\vspace{1em}

\only<3->{In underpowered study, we will be unlikely to detect true small effects. And most effects are small! \footnote{Gelman, A. and Weakliem, D. 2009. ``Of Beauty, Sex and Power.'' \textit{American Scientist} 97(4): 310--16}}

}


\frame{

\frametitle{Intuition about Power}

\begin{itemize}\itemsep1em
\item It can help to think in terms of ``standardized effect sizes''
\item Intuition: How large is the effect in standard deviations of the outcome?
	\begin{itemize}
	\item Know if effects are large or small
	\item Compare effects across studies
	\end{itemize}
\item<2-> Cohen's $d$:\\ $d = \frac{\bar{x}_1 - \bar{x}_0}{s}$, where
$s = \sqrt{\frac{(n_1 - 1)s_1^2 + (n_0 - 1)s_0^2}{n_1 + n_0 - 2}}$
\item<3-> Small: 0.2; Medium: 0.5; Large: 0.8
\end{itemize}

}

\frame{
\frametitle{Intuition about Power}

\begin{center}
\includegraphics[height=0.8\textheight, trim = 0in 0in 0in 0.5in , clip]{images/power}
\end{center}

}


\begin{frame}[fragile]

\frametitle{Power analysis in R}
\small
\begin{verbatim}
power.t.test(
  # sample size (leave blank!)
  n = ,
  
  # minimum detectable effect size
  delta = 0.4, sd = 1,
  
  # alpha and power (1-kappa)
  sig.level = 0.05, power = 0.8,
  
  # two-tailed vs. one-tailed test
  alternative = "two.sided"
)
\end{verbatim}
\end{frame}


\begin{frame}[fragile]

\frametitle{Power analysis in Stata}
\small
\begin{verbatim}
power twomeans 0, diff(0.2)

// for multiple values of 
forvalues i = 0.1 (0.1) 1.0 {
    power twomeans 0, diff(`i')
}

// using raw effect sizes and standard deviations
power twomeans 0 0.5, sd1(.5) sd2(.7)

// adjusting alpha or power
power twomeans 0, diff(0.2) alpha(0.10) power(0.7)
\end{verbatim}
\end{frame}



\begin{frame}

\frametitle{Increasing/Decreasing Power}

\begin{columns}
\begin{column}{0.5\textwidth}
\begin{block}{Increases Power}
\begin{itemize}\itemsep1em
\item Bigger sample
\item Precise measures
\item Covariates?
\end{itemize}
\end{block}
\end{column}

\begin{column}{0.5\textwidth}
\begin{block}{Decreases Power}
\begin{itemize}\itemsep1em
\item Attrition
\item Noncompliance
\item Clustering
\end{itemize}
\end{block}
\end{column}
\end{columns}

\end{frame}




\frame{}


\frame{

\frametitle{Factorial Designs}

\begin{itemize}\itemsep1em
\item The two-condition experiment is a stylized ideal

\item An experiment can have any number of conditions
	\begin{itemize}\footnotesize
	\item Up to the limits of sample size
	\item More than 8--10 conditions is typically unwieldy
	\end{itemize}

\item Three ``flavors'':
	\begin{itemize}
	\item Multiple conditions in a single factor
	\item Multiple fully \textit{crossed} factors
	\item Partially crossed (``fractional factorial'') designs
	\end{itemize}

\item Regression methods provide a generalizable tool for causal inference in such designs
\end{itemize}
}

\begin{frame}
\begin{center}
\begin{tikzpicture}[>=latex',circ/.style={draw, shape=circle, node distance=5cm, line width=1.5pt}]
    \draw (0,0) node[left, text width=3cm, align=center] (X) {Policy\\Beneficiaries};
    \draw[->] (X) -- (3,0) node[right, text width=2cm, align=center] (Y) {Policy\\Opinion};
    \draw[->] (-3,2) node[above] (Z) {Ideology} -- (X);
    \draw[->] (Z) -- (Y);
    \draw[->] (2,-3) node[below, text width=3cm, align=center] (E) {Etc.} -- (Y);
    \draw[->] (-3, -3) node[below, text width=2.5cm, align=center] (W) {Identity Salience} -- (Y);
    \draw[->] (W) -- (X);
    \draw<2->[->,thick,red] (-5,1) node[above] (T1) {Treatment 1} -- (X);
    \draw<2->[->,thick,red] (-5,-2) node[above] (T1) {Treatment 2} -- (W);
\end{tikzpicture}
\end{center}
\end{frame}


\frame{
	\frametitle{{\normalsize Example\footnote{Transue. 2007. ``Identity Salience, Identity Acceptance, and Racial Policy Attitudes: {American} National Identity as a Uniting Force.'' \textit{American Journal of Political Science} 51(1): 78--91.}}}
	

\begin{itemize}\itemsep1em
\item \only<1,3>{How close do you feel to your ethnic or racial group?}\only<2,4>{How close do you feel to other Americans?}
\item \only<1-2>{Some people have said that taxes need to be raised to take care of pressing national needs. How willing would you be to have your taxes raised to improve education in public schools?}\only<3-4>{Some people have said that taxes need to be raised to take care of pressing national needs. How willing would you be to have your taxes raised to improve educational opportunities for minorities?}
\end{itemize}
	
}


\frame{

\frametitle{2x2 Factorial Design}

\only<1>{
\begin{center}
\begin{tabular}{lr}
Condition &  \\ \midrule
Educ. for Minorities & $Y_1$ \\
Schools & $Y_0$ \\ \bottomrule
\end{tabular}
\end{center}
}

\only<2>{
\begin{center}
\begin{tabular}{lrr}
Condition & Americans & Own Race \\ \midrule
Educ. for Minorities & $Y_{1,0}$ & $Y_{1,1}$ \\
Schools & $Y_{0,0}$ & $Y_{0,1}$ \\ \bottomrule
\end{tabular}
\end{center}
}

}


\frame{

\frametitle{Two ways to \textit{parameterize} this}

Dummy variable regression (i.e., treatment--control CATEs):

$Y = \beta_0 + \beta_1 X_{0,1} + \beta_2 X_{1,0} + \beta_3 X_{1,1} + \epsilon$

\vspace{1em}

Interaction effects (i.e., treatment--treatment CATEs):

$Y = \beta_0 + \beta_1 X1_{1} + \beta_2 X2_{1} + \beta_3 X1_1 * X2_1 + \epsilon$

\vspace{1em}

Use \texttt{margins} to extract marginal effects

}


\frame<1>[label=factorialconsiderations]{

\frametitle{Considerations}

\begin{itemize}\itemsep0.5em
\item Factorial designs can quickly become unwieldy and expensive
\item<2-> Need to consider what CATEs are of theoretical interest
	\begin{itemize}
	\item Treatment--control, pairwise
	\item Treatment--treatment, pairwise
	\item Marginal effects, averaging across other factors
	\item Comparison of merged conditions
	\end{itemize}
\end{itemize}

}

\frame{

\frametitle{Probably obvious, but\dots}

\footnotesize

\begin{center}
\begin{tabular}{cccc}
Factors & Conditions per factor & Total Conditions & $n$ \\ \midrule
1 & 2 & 2 & 400 \\ 
1 & 3 & 3 & 600 \\
1 & 4 & 4 & 800 \\
2 & 2 & 4 & 800 \\
2 & 3 & 6 & 1200 \\
2 & 4 & 8 & 1600 \\
3 & 3 & 9 & 1800 \\
3 & 4 & 12 & 2400 \\
4 & 4 & 16 & 3200 \\ \bottomrule
\end{tabular}
\end{center}

{\footnotesize Assumes power to detect a relatively small effect, but no consideration of multiple comparisons.}

}

\againframe{factorialconsiderations}

% marginal effects of factors are marginal across the set of levels of the other factors; if those factors aren't complete, then external validity problem


\questions



\end{document}
