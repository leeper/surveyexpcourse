\input{preamble}

\title{Sources of Heterogeneity}

\date[]{}

\begin{document}

\frame{\titlepage}

\frame{\tableofcontents}



\section{Attention and Satisficing}
\frame{\tableofcontents[currentsection]}


% attention checks
\frame{
\frametitle{Attention Checking}

\large 
\begin{itemize}\itemsep1em
\item Online mode invites satisficing
\item Attention checking can help, but is imperfect
\end{itemize}
}

% useful for checking attention
% may have consequences for representativeness and introduce selection biases

\frame{
\frametitle{Apparent Satisficing}
\begin{itemize}\itemsep0.5em
\item Filter out respondents based on response behavior
\item Some common measures:
	\begin{itemize}
	\item ``Straightlining''
	\item Non-differentiation
	\item Acquiescence
	\item Nonresponse
	\item DK responding
	\item Speeding
	\end{itemize}
\item Difficult to detect
\item Difficult to distinguish from ``real'' responses
\end{itemize}
}

\frame{
\frametitle{Metadata/Paradata}
\begin{itemize}\itemsep1em
\item<1-> Timing
	\begin{itemize}
	\item Some survey tools will allow you to time page
	\item Make a prior rules about dropping participants for speeding
	\end{itemize}
\item<2-> Mousetracking or eyetracking
	\begin{itemize}
	\item Mousetracking is unobtrusive
	\item Eyetracking requires participants opt-in
	\end{itemize}
\item<3-> Record focus/blur browser events
\end{itemize}
}

\frame{
\frametitle{Direct Measures}
\begin{itemize}\itemsep2em
\item How closely have you been paying attention to what the questions on this survey actually mean?
\item<2-> While taking this survey, did you engage in any of the following behaviors? Please check all that apply.
	\begin{itemize}
	\item Use your mobile phone
	\item Browse the internet
	\item \dots
	\end{itemize}
\end{itemize}
}


\frame{
\frametitle{Substantive Manipulation Check}
\begin{itemize}\itemsep1em
\item Two common approaches:
	\begin{itemize}
	\item Information recall or understanding
	\item Measure level of manipulated treatment variable
	\end{itemize}
\item Risky to remove cases based on this because it is a form of conditioning on post-treatment variables
\item May be useful to consider either a mediator of effects
\end{itemize}
}

\frame{
\frametitle{Instructional Manipulation Check}

\only<2>{Do you agree or disagree with the decision to send British forces to fight ISIL in Syria? }We would like to know if you are reading the questions on this survey. If you are reading carefully, please ignore this question, do not select any answer below, and click ``next'' to proceed with the survey.\\

\vspace{1em}

\small

Strongly disagree\\
Somewhat disagree\\
Neither agree nor disagree\\
Somewhat agree\\
Strongly agree\\

}


\frame{
\frametitle{Attention Checking}

In summary\dots

\begin{itemize}\itemsep1em
\item Attention checking can be useful
\item Lots of options
\item No obvious best metric
\item Can be analytically consequential
\end{itemize}
}




\frame{}

\section[Moderators]{Moderators and Effect Heterogeneity}
\frame{\tableofcontents[currentsection]}

\subsection{Moderation}

% regression example

\frame{

If we have an hypothesis about moderation, what can we do?

\begin{itemize}
\item Best solution: manipulate the moderator
\item Next best: block on the moderator and stratify our analysis
	\begin{itemize}
	\item Estimate *Conditional* Average Treatment Effects
	\end{itemize}
\item Least best: include a treatment-by-covariate interaction in our regression model
\end{itemize}
}


\subsection{Blocking/Block Randomization}




\frame{}
\section[Factorial Designs]{Factorial and Conjoint Designs}
\frame{\tableofcontents[currentsection]}



\appendix
\frame{}

\end{document}
