\input{preamble}

\title{Session III:\\External Validity}

% external validity
% sample-related
% suto framework
%  pretreatment: Do unit history or broader context influence our results? We can measure to see if people have been pretreated, but sometimes we can't do that or can't know in advance
%  behaviors/beavioral intentions


\date[]{}

\begin{document}

\frame{\titlepage}

\frame{\tableofcontents}






\section{External Validity}
\frame{\tableofcontents[currentsection]}

\frame{

\frametitle{Think--Pair--Share}

Consider the following question:

What makes an experiment (or any research study) generalizable? What does it mean for a study's results to ``generalize''?

\begin{enumerate}
\item Write or think to yourself for 90 seconds
\item Then, discuss with the person next to you
\end{enumerate}

}


\frame{
\frametitle{``The Gold Standard''}

\begin{quote}\small
a population-based experiment uses survey sampling methods to produce a collection of experimental subjects that is representative of the target population of interest for a particular theory \dots the population represented by the sample should be representative of the population to which the researcher intends to extend his or her findings. In population-based experiments, experimental subjects are randomly assigned to conditions by the researcher
\end{quote}

{\footnotesize p2. from Mutz, Diana. 2011. \textit{Popuation-Based Survey Experiments}. Princeton University Press.\par}

}


\frame{
	\frametitle{Surveys Start with an Inference Population}
	\begin{itemize}\itemsep1em
		\item We want to speak to a population
		\item But what population is it?
			\begin{itemize}
			\item<2-> A national population?
			\item<3-> Adults in Western, industrialized democracies?
			\item<4-> All human beings?
			\end{itemize}
		\item<5-> This is rarely specified, but is important when we think about whether a sample is appropriate
	\end{itemize}
}






\frame{
	\frametitle{A Hypothetical Census}

	\begin{itemize}\itemsep2em
		\item Advantages
			\begin{itemize}
				\item<2-> Perfectly representative
				\item<2-> Sample statistics are population parameters
			\end{itemize}
		\item Disadvantages
			\begin{itemize}
				\item<3-> Costs
				\item<3-> Feasibility
				\item<3-> Need
			\end{itemize}		
	\end{itemize}
}


\frame{\frametitle{So, instead we sample!}}


\frame{
	\frametitle{Sampling Frames}
	\begin{itemize}\itemsep1em
		\item Enumeration (listing) of all units eligible for sample selection
		\item Random sample from that list
		\item<2-> Building a sampling frame
			\begin{itemize}
			\item Combine existing lists
			\item Canvass/enumerate from scratch
			\end{itemize}
		\item<3-> Concern about coverage: Does frame match population?
	\end{itemize}
}


\frame{
	\frametitle{Sample Estimates from an SRS}
	\begin{itemize}\itemsep1em
		\item Each unit in frame has equal probability of selection
		\item Sample statistics are unweighted
		\item Variances are easy to calculate
		\item Easy to calculate sample size need for a particular variance
	\end{itemize}
}

\frame{
	\frametitle{Sample mean}
	\begin{equation}
	\bar{y} = \frac{1}{n}\sum_{i=1}^{n}y_i
	\end{equation}
	where $y_i = $ value for a unit, and\\
	$n = $ sample size
	
	\begin{equation}
	SE_{\bar{y}} = \sqrt{(1-f)\frac{s^2}{n}}
	\end{equation}
	where $f = $ proportion of population sampled,\\
	$s^2 = $ sample element variance, and\\
	$n = $ sample size
}

\frame{
	\frametitle{Estimating sample size}

\small

	If all we cared about was a single proportion:

	\begin{equation}
		Var(p) = (1-f)\frac{p(1-p)}{n-1}
	\end{equation}
	
	Given a large population:	
	\begin{equation}
		Var(p) = \frac{p(1-p)}{n-1}
	\end{equation}
	
	Need to solve the above for $n$.
	\begin{equation}
	\only<2->{n = \frac{p(1-p)}{v(p)} = \frac{p(1-p)}{SE^2}}
	\end{equation}
}

\frame{
	\frametitle{Estimating sample size}
    Determining sample size requires:
    	\begin{itemize}
    		\item A possible value of $p$
    		\item A desired precision (standard error)
    	\end{itemize}
	\vspace{1em}
	If support for each coalition is evenly matched ($p = 0.5$):
	\begin{equation}
	n = \frac{0.5(1-0.5)}{SE^2} = \frac{0.25}{SE^2}
	\end{equation}
}

\frame{
	\frametitle{Estimating sample size}
	What precision (margin of error) do we want?
	\begin{itemize}
		\item +/- 2 percentage points: $SE = 0.01$
			\begin{equation}
			n = \frac{0.25}{0.01^2} = \frac{0.25}{0.0001} = 2500
			\end{equation}
		\item<2-> +/- 5 percentage points: $SE = 0.025$
			\begin{equation}
			n = \frac{0.25}{0.000625} = 400
			\end{equation}
		\item<3-> +/- 0.5 percentage points: $SE = 0.0025$
			\begin{equation}
			n = \frac{0.25}{0.00000625} = 40,000
			\end{equation}
	\end{itemize}
}



\frame{

\frametitle{Sampling Considerations\dots}

\begin{itemize}\itemsep1em
\item<2-> More complex designs possible, all based on each unit having a \textit{known, non-zero} probability of being sampled
	\begin{itemize}
	\item Stratified sampling can produce lower variances
	\end{itemize}
\item<3-> Random sampling ensures that samples are, \textit{in expectation}, representative of the population \textit{in all respects}
	\begin{itemize}
	\item Demographics
	\item Psychological traits
	\item Covariances
	\item Potential outcomes
	\end{itemize}
\end{itemize}
}


\frame{}

\bgroup
\setbeamercolor{background canvas}{bg=blue!20!white}
\setbeamertemplate{navigation symbols}{}
\begin{frame}[plain]{}
\frametitle{Representativeness}

What does it mean for a sample to be representative?

\begin{itemize}
	\item<2-> Census
	\item<3-> SRS (or more complex) design
	\item<4-> Quota sampling (common prior to the 1940s)
	\item<5-> Demographics match population
	\item<6-> Others?
\end{itemize}

\vspace{1em}

\onslide<7->{Which of these matter?}
\end{frame}
\egroup

% what kinds of representativeness do we care about?



\frame{

\frametitle{{\large Combining Probability Sampling and\\ Experimental Design}}

\begin{itemize}\itemsep1em
\item Sample is representative of population in every respect (in expectation)
\item Sample Average Treatment Effect (SATE) is the average of the sample's individual-level treatment effects
	\begin{itemize}
	\item Unbiased estimate of PATE
	\item Not necessarily any unit's individual treatment effect
	\item Blocking might reduce variance
	\end{itemize}
\item Says nothing about effect heterogeneity
	\begin{itemize}
	\item Design is optimized for estimating SATE
	\end{itemize}
\end{itemize}
}



\frame{

Credibility of all of this is based on \textit{design} only

\vspace{1em}

\onslide<2->{Sampling aspect only works in a world of perfect coverage and no response bias}

}

\frame<1-3>[label=myview]{
\frametitle{My View}

100\% design-based inference does not exist

\vspace{1em}

\begin{itemize}\itemsep0.75em
\item<2-> All survey designs involve reweighting adjustments
\item<3-> Representativeness is a more complex issue than demographic comparisons
\item<4-> Randomization gives us clear causal inference about a \textit{local} effect
	\begin{itemize}
	\item I would always sacrifice representativeness for clarity of causal inference
	\item Focus on figuring out the nature of the \textit{localness}
	\end{itemize}
\end{itemize}
}




\frame{
\frametitle{My Own Research}
\footnotesize
\begin{tabular}{ l r r r r r r r} \toprule\toprule
   & {\bf GfK} & {\bf Poll} & {\bf Student} & {\bf Staff}  & {\bf MTurk} & {\bf Ads} & {\bf ANES}\\ \midrule
{\bf Dem. (\%)}      & 51.3 & 86.1 & 75.7 & 66.4 & 62.1 & 72.1 & 46.2 \\  
{\bf Rep. (\%)}    & 46.0 &   7.7 & 17.8 & 16.4 & 20.3 &  14.7 & 39.3\\ 
{\bf Lib. (\%)}          & 27.8 & 75.4 & 68.5 & 62.7 & 60.4 & 66.2& 23.8\\
{\bf Con. (\%)} & 35.3 & 9.4 & 14.7 & 19.8 & 19.1 & 17.7& 36.1\\
{\bf Fem. (\%)}         & 51.1 & 60.8 & 56.4 & 50.8 & 41.7 & 65.3 & 51.9 \\
{\bf White (\%)}           & 77.9 & 67.6 & 62.9 & 60.2 & 76.0 & 53.8 & 80.4\\
{\bf Age}          & 49.4 & 40-49 & 18-24 & 25-34 & 25-34 & 25-34& 50-54 \\ 
{\bf Interest}       & 2.8 & 3.5 & 3.2 & 2.8 & 2.7 & 3.0 & 3.0\\ \midrule
{\bf N}                         & 593 & 741 & 299 & 128 & 1024 & 80&  -- \\ \bottomrule\bottomrule
\end{tabular}

\vspace{1em}

{\footnotesize Mullinix et al. In press. ``The Generalizability of Survey Experiments.'' \textit{Journal of Experimental Political Science}.}
}

\frame{
\begin{center}
\includegraphics[height=.95\textheight]{images/mullinix3}
\end{center}
}


\frame{
\begin{center}
\includegraphics[width=1.1\textheight, trim=0in 0in 0in 0.5in, clip]{images/mullinix1}
\end{center}
}

\againframe<3-4>{myview}

\frame{
\frametitle{SUTO Framework}
\begin{itemize}\itemsep1em
\item Cronbach (1986) talks about generalizability in terms of UTO
\item Shadish, Cook, and Campbell (2001) speak similarly of:
	\begin{itemize}
	\item \textbf{S}ettings
	\item \textbf{U}nits
	\item \textbf{T}reatments
	\item \textbf{O}utcomes
	\end{itemize}
\item External validity depends on all of these things
\end{itemize}
}


\frame{
\begin{columns}[t]
\begin{column}{0.5\textwidth}
	\begin{block}{Population}
		\begin{itemize}
		\item Setting
		\item Units
		\item Treatments
		\item Outcomes
		\end{itemize}
	\end{block}
\end{column}
\begin{column}{0.5\textwidth}
	\begin{block}{Your Study}
		\begin{itemize}
		\item Setting
		\item Units
		\item Treatments
		\item Outcomes
		\end{itemize}
	\end{block}
\end{column}
\end{columns}

\vspace{1em}

\only<2->{In your study, how do these correspond?\\}
\only<3->{\hspace{5.7em} how do these differ?\\}
\only<4->{\hspace{5.7em} do these differences matter?\\}

}


\frame{
\frametitle{Common Differences}
\begin{itemize}\itemsep1em
\item Most common thing to focus on is demographic representativeness
	\begin{itemize}
	\item Sears (1986): ``students aren't real people''
	\item \href{http://www.slate.com/articles/health_and_science/science/2013/05/weird_psychology_social_science_researchers_rely_too_much_on_western_college.html}{Western, educated, industrialized, rich, democratic (WEIRD) psychology participants}
	\end{itemize}
\item<2-> But do those characteristics actually matter?
\item<3-> Shadish, Cook, and Campbell tell us to think about:
	\begin{itemize}
	\item Surface similarities
	\item Ruling out irrelevancies
	\item Making discriminations
	\item Interpolation/extrapolation
	\end{itemize}
\end{itemize}
}


\frame{
\frametitle{Focus on effect heterogeneity}
\begin{itemize}\itemsep1em
\item Think about and make an evidence-based argument for why you think there are (or are not) heterogeneous effects
\item<2-> If you think there is heterogeneity, then we probably do not care about the SATE anyway
	\begin{itemize}
	\item Focus on CATE instead
	\item $E[Y_{1i} | X = 1, Z=z] - E[Y_{0i} | X = 0, Z=z]$
	\end{itemize}
\item<3-> Two formal analytic strategies
	\begin{itemize}
	\item Regression with large number of interactions
	\item Bayesian Additive Regression Trees
	\end{itemize}
\item<4-> But remember: you have to convince reviewers!
\end{itemize}
}



\frame{
\frametitle{BART}
\begin{itemize}\itemsep1em
\item Estimate \textit{conditional average treatment effects}
\item BART is essentially an ensemble machine learning method
\item Iteratively split a sample into more and more homogeneous groups until some threshold is reached using binary (cutpoint) decisions
\item Repeat this a bunch of times, aggregating across results
\end{itemize}
}

\frame{
\begin{center}
\includegraphics[width=\textwidth]{images/greenkern1}
\end{center}
{\footnotesize Green and Kern. 2012. ``Modeling Heterogeneous Treatment Effects in Survey Experiments with Bayesian Additive Regression Trees.'' \textit{Public Opinion Quarterly}.\par}
}

\frame{
\begin{center}
\includegraphics[height=.8\textheight]{images/greenkern2}
\end{center}
{\footnotesize Green and Kern. 2012. ``Modeling Heterogeneous Treatment Effects in Survey Experiments with Bayesian Additive Regression Trees.'' \textit{Public Opinion Quarterly}.\par}
}


\frame{

\frametitle{Stratification/Blocking}

As soon as we care about heterogeneous effects, it makes sense to stratify and block on factors that might moderate the treatment effect.

\vspace{1em}

\onslide<2->{As soon as we identify all sources of heterogeneity, it doesn't matter what sample we use because effects are \textit{by definition} homogeneous within such strata.}

\vspace{1em}

\onslide<3->{But, we never know when we've reached that point!}

}



\frame{

If we acknowledge and start thinking about effect heterogeneity, does this mean we can use any convenient group of participants as if they were probability samples?

\vspace{1em}

No. Of course not.
}


% defining your convenience sample
\frame{
	\frametitle{Not All Convenience Samples Are Alike}
	\begin{itemize}\itemsep1em
		\item<1-> Different types:
			\begin{itemize}
				\item<2-> Passive/opt-in/``river sampling''
				\item<3-> Sample of convenience (not a sample per se)
					\begin{itemize}
					\item Email list
					\item Snowball sample
					\item Respondent-driven Sampling
					\item Students
					\item Crowdsourcing
					\end{itemize}
			\end{itemize}
		\item<4-> Differ in numerous ways
			\begin{itemize}
			\item Cost
			\item ``Experience''
			\item Attentiveness
			\item Demographics
			\end{itemize}
	\end{itemize}
}

\frame{
\frametitle{Costs per participant}

From one of my studies:\\

\vspace{1em}

\centering
\begin{tabular}{l r r r}
Sample	& Cost	& n	& Cost/participant\\ \midrule
National	& \$13200	& 593	& \$22.26\\
Exit Poll	& \$3000	& 741	& \$4.05\\
Students	& \$0	& 299	& \$0\\
Staff		& \$1280	& 128	& \$10.00\\
MTurk	& \$550	& 1024	& \$0.54\\
Ads		& \$636	& 80		& \$7.95\\
\bottomrule
\end{tabular}
}

\frame{
\frametitle{Participant Experience}

\begin{itemize}\itemsep1em
\item A lot of growing concern about experience
\item Larger literature on ``panel conditioning''
	\begin{itemize}
	\item Inconclusive evidence
	\end{itemize}
\item<3-> Some numbers:
	\begin{itemize}
	\item<3-> MTurk workers are doing 100+ studies per month
	\item<4-> Numbers are the same for YouGov panelists
	\end{itemize}
\end{itemize}

}



% reweighting; pscore methods
\frame{
\frametitle{Reweighting}
\begin{itemize}\itemsep1em
\item It may be possible to \textit{reweight} convenience sample data to match a population
\item Any method for this is ``model-based'' (rather than ``design-based'')
\item Not widely used or evaluated (yet)
\item All techniques build on the idea of stratification
\end{itemize}
}


\frame{
	\frametitle{Overview of Stratification}
	\begin{enumerate}\itemsep0.5em
		\item Define population
		\item Construct a sampling frame
		\item Identify variables we already know about units in the sampling frame
		\item Stratify sampling frame based on these characteristics
		\item Collect an SRS (of some size) within each stratum
		\item Aggregate our results
	\end{enumerate}
}

\frame{
	\frametitle{Estimates from a stratified sample}
	\begin{itemize}\itemsep0.5em
		\item Within-strata estimates are calculated just like an SRS
		\item Within-strata variances are calculated just like an SRS
		\vspace{1em}
		\item Sample-level estimates are weighted averages of stratum-specific estimates
		\item Sample-level variances are weighted averages of strataum-specific variances
	\end{itemize}
}


\frame{
	\frametitle{Post-Stratification}
	\begin{itemize}\itemsep1em
		\item Used to correct for nonresponse, coverage errors, and sampling errors
		\item<2-> Reweight sample data to match population distributions
			\begin{itemize}
				\item Divide sample and population into strata
				\item Weight units in each stratum so that the weighted sample stratum contains the same proportion of units as the population stratum does
			\end{itemize}
		\item<3-> There are numerous other related techniques
	\end{itemize}
}

\frame{
	\frametitle{Post-Stratification: Example}
	\begin{itemize}\itemsep1em
		\item Imagine our sample ends up skewed on immigration status and gender relative to the population\\
		\vspace{1em}
		\small
		\begin{tabular}{lrrlr}
			\hline
			Group             & Pop. & Sample & Rep.                &              Weight \\ \hline
			Native-born, Female    &  .45 &     .5 & \onslide<2->{Over}  & \onslide<3->{0.900} \\
			Native-born, Male      &  .45 &     .4 & \onslide<2->{Under} & \onslide<4->{1.125} \\
			Immigrant, Female &  .05 &    .07 & \onslide<2->{Over}  & \onslide<4->{0.714} \\
			Immigrant, Male   &  .05 &    .03 & \onslide<2->{Under} & \onslide<4->{1.667}\\  \hline
		\end{tabular}
		\item PS weight is just $w_{ps} = N_l / n_l$
	\end{itemize}
}


\frame{
	\frametitle{Post-Stratification}
	\begin{itemize}\itemsep1em
		\item This is the basis for inference in non-probability samples
		\begin{itemize}
			\item \textit{Demographic} representativeness
		\end{itemize}
		\item Online panels will reweight sample based on age, sex, education, etc.
		\item Purely design-based surveys are increasingly rare
	\end{itemize}
}


\frame{
\frametitle{The Xbox Study}

\begin{center}
\includegraphics[width=\textwidth]{images/wangetal2}
\end{center}

{\footnotesize Wang et al. 2015. ``Forecasting elections with non-representative polls.'' \textit{International Journal of Forecasting}.\par}

}


\frame{
\frametitle{The Xbox Study}

\begin{center}
\includegraphics[width=\textwidth]{images/wangetal1}
\end{center}

\vspace{1em}

{\footnotesize Wang et al. 2015. ``Forecasting elections with non-representative polls.'' \textit{International Journal of Forecasting}.\par}

}



\frame{
\begin{center}
\includegraphics[width=.85\textwidth]{images/mullinix1}
\end{center}
\vspace{0.5em}
{\footnotesize Mullinix et al. In press. ``The Generalizability of Survey Experiments.'' \textit{Journal of Experimental Political Science}.\par}
}

\frame{
\begin{center}
\includegraphics[width=.8\textheight]{images/mullinix2}
\end{center}
}



\frame{
\frametitle{Propensity Score Approach}
\begin{enumerate}
\item Define a target population to which sample inference is intended to generalize
\item Estimate a propensity score model
	\begin{itemize}
	\item Pool experimental samples and target population units
	\item Predict membership of all target and sample units in the experimental sample
	\end{itemize}
\item Using fitted logits, divide the population and sample into strata
	\begin{itemize}
	\item Number of strata is commonly 5 (Cochran, 1968)
	\end{itemize}
\item Estimate stratum-specific ATE
\item Calculate weighted average of stratum-level estimates
\end{enumerate}
}


\frame{
\frametitle{Propensity Score Approach}

Target population average treatment effect:
\begin{equation}
\sum_{v=1}^{5} p(v)T(v)
\end{equation}
where $p(v)$ is the proportion of the target population in a given stratum, $v$, and $T(v)$ is the estimated effect from stratum $v$ of the experimental sample
}


\frame{
\frametitle{Propensity Score Approach}

Effect variance:
\begin{equation}
\sum_{v=1}^{5} p(v)^2 V(v),
\end{equation}
where $V(v)$ is the variance of the estimated experimental sample effect for stratum $v$
}



\frame{
\frametitle{Propensity Score Subclassification Estimator}
\begin{columns}
\column{\dimexpr\paperwidth-15pt}
\scriptsize
\begin{tabular}{rrrrrrrrr} \toprule
& \multicolumn{2}{c}{Weights} & \multicolumn{6}{c}{Estimates}\\
Stratum & Nat'l & Sample & Loan & DREAM (Pro) & DREAM (Con) & Rally (All) \\% & Rally (Distant) & Rally (Local) \\ 
  \midrule
  1 & 0.20 & 0.83 & 0.94 (0.08) & 0.06 (0.11) & -0.22 (0.12) & 0.74 (0.10) \\% & 0.77 (0.15) & 0.72 (0.15) \\ 
  2 & 0.20 & 0.11 & 0.99 (0.26) & 0.22 (0.37) & -0.28 (0.36) & 0.77 (0.29) \\% & 0.85 (0.39) & 0.70 (0.42) \\ 
  3 & 0.20 & 0.04 & 1.28 (0.43) & -0.61 (0.58) & -1.76 (0.54) & 1.00 (0.45) \\% & 0.75 (0.64) & 1.26 (0.64) \\ 
  4 & 0.20 & 0.01 & 1.99 (0.73) & 0.29 (1.12) & 0.56 (0.89) & 1.44 (0.79) \\% & 3.13 (1.95) & 1.54 (0.63) \\ 
  5 & 0.20 & 0.00 &  &  &  &  &  &  \\ \midrule
  Sample & - & - & 1.04 (0.30) & -0.01 (0.44) & -0.34 (0.38) & 0.79 (0.33) \\% & 1.05 (0.60) & 0.86 (0.37) \\ 
  Nat'l & - & - & 1.14 (0.18) & 0.02 (0.22) & -0.94 (0.23) & 0.94 (0.19) \\% & 1.01 (0.26) & 0.93 (0.27) \\ 
   \bottomrule
\end{tabular}
\end{columns}
}


\frame{
\textbf{So does reweighting solve everything forever?}

\begin{itemize}\itemsep1em
\item<2-> Need well-defined target population
	\begin{itemize}
	\item and detailed covariate data
	\item and large stratum sizes
	\end{itemize}
\item<3-> Purely model-based, so only as good as the model
	\begin{itemize}
	\item What unobservables might be hiding bias?
	\item What reweighting might worse bias?
	\end{itemize}
\item<4-> Non-coverage is a potentially huge problem
\item<5-> Not well-tested on experimental data
\end{itemize}

}


% eligibility; duplication; recontacts
\frame{
\frametitle{My Advice}
\begin{itemize}\itemsep1em
\item Only work with enumerated populations
	\begin{itemize}
	\item Each unit is uniquely identifiable
	\end{itemize}
\item<2-> Without this, you risk many things:
	\begin{itemize}
	\item Ambiguous eligibility
	\item Retakes, treatment crossover
	\item No way to evaluate response rates/bias
	\end{itemize}
\item<3-> Know something about your sample
	\begin{itemize}
	\item How does it differ from your target of inference?
	\item What theories or evidence would suggest those differences should matter?
	\item What can you do to adjust or control for those \textit{consequential} differences?
	\end{itemize}
\end{itemize}
}





\appendix
\frame{}

\end{document}
